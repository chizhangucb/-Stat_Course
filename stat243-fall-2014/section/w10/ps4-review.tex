\documentclass{article}

%% for inline R code: if the inline code is not correctly parsed, you will see a message
\newcommand{\rinline}[1]{SOMETHING WRONG WITH knitr}
%% begin.rcode setup, include=FALSE, cache=FALSE
% opts_chunk$set(fig.path='figure/latex-', cache.path='cache/latex-')
% read_chunk('ps4.R')
%% end.rcode

\title{Code review for problem set 4}
\date{October 27, 2014}
\author{Jarrod Millman\\ Statistics 243\\ UC Berkeley}

\begin{document}

\maketitle

For section today, I will ask you to form a few small ($<5$ people) groups.
In your groups, you should discuss the following questions and then I will
randomly ask one of the groups to informally summarize some of the ideas
they discussed.  When presenting feel free to draw on the board or ask
me and/or the other students questions about parts you are unclear about.

\begin{enumerate}
\setcounter{enumi}{3}

\item Random walks

For question 4, you were asked to implement a random walk function. Consider
the following code snippet from Chris' example solution.

%% begin.rcode  q4.sol, eval=FALSE
%% end.rcode

Look through this implementation and make sure you understand how it works.
In particular, make sure you can explain how this code is generating the
random steps.  Here are some specific questions you may wish to consider:

\begin{itemize}
\item What purpose does \texttt{cumsum} serve?
\item If \texttt{cumsum} didn't exist, how would you modify this code snippet
 so that it still worked?  Which of these implementations makes the code easier
 to understand and debug?  Are there any implications for speed and memory use?
\item R has many functions for drawing random samples.  Which of these other
 functions might be useful if \texttt{sample} didn't exist?  Would you prefer
 using one of these other functions over \texttt{sample}?  For example, are
 any of them more or less similar to how you think about the underlying mathematical
 model?  Are there any implications for speed and memory use?
\end{itemize}

\item Memory use

For extra credit on question 5, you were asked to rewrite a function in
order to reduce its memory use.  For this exercise, please consider the following
portion of the original code:

%% begin.rcode  q5.orig, eval=FALSE
%% end.rcode

As well as the corresponding part of Chris' example solution:

%% begin.rcode  q5.sol1, eval=FALSE
%% end.rcode

Make sure you can explain why this works and why it is uses less memory.
Are there any trade-offs (perhaps in terms of speed, generality, or readability)
that were made for the sake of reducing memory use?

Now consider the following alternative solutions:

\begin{enumerate}

\item

%% begin.rcode  q5.sol2, eval=FALSE
%% end.rcode

\item

%% begin.rcode  q5.sol3, eval=FALSE
%% end.rcode

\item

%% begin.rcode  q5.sol4, eval=FALSE
%% end.rcode

\end{enumerate}

Do they all produce the same result?
Are there similarities among the different solutions?
Do these solutions make different trade-offs from each other or from
Chris' example solution?
When looking over the different solutions are there some that you prefer?
If so, why?

\item Linear algebra and loops

Finally, for question 6, you were asked to rewrite a function to minimize
its computation time.  There were several things that you could do, but the
biggest reduction came from rewriting this portion of the code:

%% begin.rcode  q6.orig, eval=FALSE
%% end.rcode

Here is a possible solution:

%% begin.rcode  q6.sol, eval=FALSE
%% end.rcode

Does this perform the same computation?  Is it more or less readable than
the original?  Is one of the two implementations easier to understand?  Why?

In the proposed solution, what other commands could you use to replace the
body of the loop without adding or changing the loop?  For example, Chris'
example solution used \texttt{outer(theta.old[, z],theta.old[, z])}.  Are
there some that you prefer?  Why?  For example, are some of them easier for
you to read and understand?  Are some closer to the underlying mathematical
expression that you would use?  Are they all equivalent in terms of space
and time complexity?  

Finally, consider the following alternative loop:

%% begin.rcode  q6.alt1, eval=FALSE
%% end.rcode

Does it perform the same computation?  Explain.
Assuming it is correct, are there any cases when this loop might be
preferred over the other proposed loop (or \emph{vice versa})?


\end{enumerate}

\end{document}

